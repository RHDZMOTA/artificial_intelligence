\documentclass[11p]{report}
\usepackage{amsmath}
\usepackage{fancyhdr}

\pagestyle{fancy}
%\parident 0em
\parskip 2ex

%\setlenght{\parskip}{\baselineskip}
%\setlenght{\parindent}{24pt}
%\setlenght{\baselineskip}{24pt}

\begin{document}

\parindent 0em

\hspace{0.7cm}

\title{\textbf{Examen I}}
\author{Rodrigo Hern\'andez Mota (if 693056)}

\maketitle

\section{I}
El algor\'itmo gen\'etico permite optimizar una funci\'on de la forma $F(X)$, en donde $X = {x_1,...,x_n}$ con un m\'etodo heur\'istico que pretende simular el sistema de recombinaci\'on gen\'etico biol\'ogico. 
\subsection{Inicializaci\'on}
Dada una regi\'on de busqueda se inicializa la poblaci\'on de 'm' pobladores. Los pobladores son distintos valores $X$ para los cuales la funci\'on puede ser evaluada.
\subsection{Selecci\'on}
Se eval\'ua el desempeño de cada poblador mediante el valor resultante de la funci'on $F$. Dependiendo si el objetivo es maximizar o minimizar, se seleccionan los pobladores m\'as aptos. Se deber\'an escoger al menos dos para asegurar la recombinaci\'on gen\'etica. 
\subsection{Cruzamiento}
Habiendo seleccionado los mejores pobladores, se utiliza su c\'odigo gen\'etico para generar nuevos pobladores (nueva generaci\'on). En esta generaci\'on estan los mejores pobladores anteriores y los nuevos resultantes del cruzamiento.

Existen diferentes m\'etodos de cruzamiento.   
\subsection{Mutaci\'on}
Para evitar la degeneraci\'on de la poblaci\'on se establece un factor de mutaci\'on. Parte de la nueva generaci\'on sufre alteraciones en su c\'odigo gen\'etico (generalmente representado como un numero en binario). 

Los pasos a partir de la inicializaci\'on son repetidos de forma iterativa. De esta forma la poblaci\'on mejora (o intenta mejorar) su desempeño generaci\'on tras generaci\'on. 

\section{II}
Sea $F$ una funci\'on de $x_1, x_2$ en donde $min F(x_1, x_2)$ se encuentra en los intervaloes $1 \leq x_1 2$ y $-2 \leq x_2 \leq 2$, entonces si se requiere que el algor\'itmo pueda buscar en intervalos de $2.44x10^{-4}$, el n\'umero de bits necesarios son: 14.

Declaramos $\Delta$ como una variable para determinar el "salto" o cambio que se dar\'a debido a la divisi\'on o discretizaci\'on  (determinado por los bits) del espacio de b\'usqueda real. 

$$
\Delta = \frac{ls - li}{2^n - 1}
$$

En donde $ls, li$ son los l\'imites superior e inferior y $n$ es el numero de bits. 
Entonces calculamos delta para cada variable:

$$
\Delta_1 = \frac{1}{m}
$$

$$
\Delta_2 = \frac{4}{m}
$$

Debido a que $1/m < 4/m$ entonces para asegurar que el algor\'itmo pueda buscar en (al menos) en el intervalo deseado; se calcula $m$ de la siguente forma:

$$
m = 4 * \Delta_2^{-1} = \frac{4}{2.44x10^{-4}} = 16393.4426
$$

Despejando $n$ y sustituyendo $m$ en $2^n - 1 = m$ tenemos:

$$
n = \log_2{m+1} = \log_2{16394.4426} = 14.000919
$$

Debido a que el "numero de bits" en pr\'actica debe ser entero; se rendondea al enterno más cercano;

$$
n = 14
$$

\section{III}
Sea la $F$ la funci\'on por partes:

$$
\begin{displaymath}
  F(x_1,x_2) = \left\{
    \begin{array}{ll}
      x_1^2 + x_2^2 & : x_1 + x_2 \geq 1 \\
      -10(x_1 + x_2)+20 & : x_1 + x_2 < 1
    \end{array}
  \right.
\end{displaymath}
$$

Considerando la reestricc\'on de ${x_1,x_2} > 1$ entonces el valor de $X$  que minimiza $F$ es:

$$
\begin{array}{ll}
x_1 = 0.4983 \\
x_2 = 0.5101 \\
\end{array}

Generando un $F = 0.500$. 

\section{IV}

Utilizando la descripción y datos del problema, se propone una función:

$$
F = f(x_1, x_2, x_3, x_4, x_5)
$$

Sujeto a las reestricciones:

$$
\begin{array}
x_1 \geq 48 \\
x_1 + x_2 \geq 79 \\
x_1 + x_2 \geq 65 \\
x_1 + x_2 + x_3 \geq 87 \\
x_2 + x_3 \geq 64 \\
x_3 + x_4 \geq 73 \\
x_3 + x_4 \geq 82 \\
x_4 \geq 43 \\
x_4 + x_5 \geq 52 \\
x_5 \geq 15
\end{array}
$$

En donde $F$ representa el costo de la empresa (pago a empleados), ${x1,\ldots,x5}$ son el numero de empleados por turno $i = 1,2,...$ y las reestricciones son el numero m\'inimo de agentes necesarios.

Resultado (n\'umero de empleados por turno): 

$$
\begin{array}
x_1 = 50\\
x_2 = 29\\
x_3 = 39\\
x_4 = 43\\
x_5 = 16\\
\end{array}
$$

Generando un costo (aprox.) de $30676.5198449$ unidades monetarias.

\end{document}
