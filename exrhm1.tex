\documentclass[11p]{report}
\usepackage{fancyhdr}

\pagestyle{headings}

\setlenght{\parskip}{\baselineskip}
\setlenght{\parindent}{24pt}
%\setlenght{\baselineskip}{24pt}

\begin{document}

\title{Examen I}
\author{Rodrigo Hern\'andez Mota}

\maketitle

\section{I}
El algor\'itmo gen\'etico permite optimizar una funci\'on con un m\'etodo heur\'istico...
\subsection{Inicializaci\'on}
\subsection{Selecci\'on}
\subsection{Cruzamiento}
\subsection{Mutaci\'on}


\section{II}
Sea $F$ una funci\'on de $x_1, x_2$ en donde $min F(x_1, x_2)$ se encuentra en los intervaloes $1 \leq x_1 2$ y $-2 \leq x_2 \leq 2$, entonces si se requiere que el algor\'itmo pueda buscar en intervalos de $2.44x10^{-4}$, el n\'umero de bits necesarios son: 14.

Declaramos $\Delta$ como una variable para determinar el "salto" que se dar\'a cuando se pasa de un entero (determinado por los bits) al espacio de b\'usqueda real. 

$$
\Delta = \frac{ls - li}{2^n - 1}
$$

En donde $ls, li$ son los l\'imites superior e inferior y $n$ es el numero de bits. 
Entonces calculamos delta para cada variable:

$$
\Delta_1 = \frac{1}{m}
$$

$$
\Delta_2 = \frac{4}{m}
$$

Debido a que $1/m < 4/m$ entonces para asegurar que el algor\'itmo pueda buscar en (al menos) en el intervalo deseado; se calcula $m$ de la siguente forma:

$$
m = 4 * \Delta_2^{-1} = \frac{4}{2.44x10^{-4}} = 16393.4426
$$

Despejando $n$ y sustituyendo $m$ en $2^n - 1 = m$ tenemos:

$$
n = \log_2{m+1} = \log_2{16394.4426} = 14.000919
$$

Debido a que el "numero de bits" en pr\'actica debe ser entero; se rendondea al enterno más cercano;

$$
n = 14
$$

\section{III}
Sea la $F$ la función por partes:

$$
\begin{displaymath}
  F(x_1,x_2) = \left\{
    \begin{array}{lr}
      x_1^2 + x_2^2 & : x_1 + x_2 \geq 1 \\
      -10(x_1 + x_2)+20 & : x_1 + x_2 < 1
    \end{array}
  \right.
\end{displaymath}
$$

Considerando la reestricc\'on de ${x_1,x_2} > 1$ entonces el valor de $X$  que minimiza $F$ es:

$$
\begin{array}{ll}
x_1 = 0.4983 \\
x_2 = 0.5101 \\
\end{array}

Generando un $F = 0.500$. 

\section{IV}

\end{document}
