\documentclass[11p]{report}
\usepackage{fancyhdr}

\pagestyle{headings}
\setlenght{\baselineskip}{24pt}

\begin{document}

\title{Examen I}
\author{Rodrigo Hern\'andez Mota}

\maketitle

\section{I}
El algor\'itmo gen\'etico permite optimizar una funci\'on con un m\'etodo heur\'istico...
\subsection{Inicializaci\'on}
\subsection{Selecci\'on}
\subsection{Cruzamiento}
\subsection{Mutaci\'on}


\section{II}
Sea $F$ una funci\'on de $x_1, x_2$ en donde $min F(x_1, x_2)$ se encuentra en los intervaloes $1 \leq x_1 2$ y $-2 \leq x_2 \leq 2$, entonces si se requiere que el algor\'itmo pueda buscar en intervalos de $2.44x10^{-4}$, el n\'umero de bits requerido es: 14.


\section{III}

\section{IV}

\end{document}
